\documentclass{article}
\usepackage[utf8]{inputenc}
\usepackage{amsmath}
\usepackage{geometry}
% \usepackage{indentfirst}
\title{BIM3008-Assignment1}
\author{Junyang Deng (120090791)}
\date{\today}
\geometry{left=3cm,right=3cm,top=1cm,bottom=1cm}
\setlength{\parindent}{0cm}
\begin{document} 
\maketitle
\section*{Question 1}
Please conduct a literature survey on the experimental principles and the performance of the Nucleic Acids Test or Antigen Test for COVID-19 SARS-CoV-2 infections. Briefly describe the principles and the predictive model performance using some measures, including precision, sensitivity, and specificity.
\subsubsection*{Answer}
The principle of Nucleic Acids Test: The test is based on a real-time fluorescent PCR platform. With RNA transcription, polymerase chain reaction and TaqMan probe technology to target the highly conserved regions of the ORF1ab gene and N gene of SARS-CoV-2. 

According to Jarrom et al. (2022), the sensitivity of a reverse-transcriptase PCR test is estimated to be 87.8\%.  The specificity ranges from 95\% to 100\%. However, the sensitivity is subject to the way of collecting samples.

Reference:
\begin{itemize}
    \item Jarrom D, Elston L, Washington J, et al. (2022). Effectiveness of tests to detect the presence of SARS-CoV-2 virus, and antibodies to SARS-CoV-2, to inform COVID-19 diagnosis: a rapid systematic review. BMJ Evidence-Based Medicine 2022;27:33-45.
\end{itemize}


% References:
% https://ebm.bmj.com/content/27/1/33
% https://www.fda.gov/media/138685/download
% Van Walle, I., Leitmeyer, K., \& Broberg, E. K. (2021). Meta-analysis of the clinical performance of commercial SARS-CoV-2 nucleic acid and antibody tests up to 22 August 2020. Eurosurveillance, 26(45), 2001675.



\section*{Question 2}
Let us say we are given the task of building an automated taxi. Define the constraints. What are the inputs? What is the output? How can we communicate with the passenger? Do we need to communicate with the other automated taxis, that is, do we need a “language”? (Question 4 of Chapter One \#41)
\subsubsection*{Answer}

Constraints are: \\
(1) taxis should run on roads,\\ 
(2) taxis should not hit other cars, \\
(3) taxis should obey the traffic rules (i.e. stop when traffic lights turn red), \\
(4) taxis should not hit pedestrians.\\

\noindent Inputs are: \\(1) the traffic conditions of nearby region, \\(2) the location of surrounding objects, like pedestrians, cars, signs.\\

\noindent Outputs are: Direction and speed of taxis.\\

\noindent Ways to communicate with the passanger: figures, video and audio.

\section*{Question 3}
\subsection*{Problem 6 of Chapter 3}
Somebody tosses a fair coin and if the result is heads, you get nothing; otherwise, you get \$5. How much would you pay to play this game? What if the win is \$500 instead of \$5?

\subsubsection*{Answer}
There are two outcomes of tossing a fair coin, with each of them have the probability of 1/2. We can draw the probability table:
\begin{table}[h]
    \begin{tabular}{|l|l|l|}
    \hline
    Events & Head  & Tail  \\ \hline
    Money (X)  & X=5     & X=0     \\ \hline
    P      & 1/2 & 1/2 \\ \hline
    \end{tabular}
    \end{table}
\\The expectation is $\mathbf{E}(x) = 5\times 1/2 + 0 \times 1/2 = 5/2$. This means I won't pay more than 2.5\$ to play this game.
\\If the win is 500\$ instead of 5\$, the expectation is $\mathbf{E}(x) = 500\times 1/2 + 0 \times 1/2 = 250$. However, I won't take the risk of paying 250 dollars for this game since the variance is very high. 

\subsection*{Problem 11 of Chapter 3}
Show example transaction data where for the rule X → Y:\\
(a) Both support and confidence are high;\\
(b) Support is high and confidence is low;\\
(c) Support is low and confidence is high;\\ 
(d) Both support and confidence are low.
\subsubsection*{Answer}
(a) Pencil and rubber: many people buy pencils and rubbers together, so the support is high; Given that a person buys pencils, he may also buy rubbers as well, so the confidence is high too.\\
(b) Tissue and fruit: many people will buy tissue and fruit together, since they are all neccessies, so the support is high. However, when a person buys tissue, he may or may not buys fruit, so the confidence is low.\\
(c) A book about toxiology and a book about pharmacology: Not many people buy these two books, so the support is low. But a person who buys a toxiology book (who may be a medical student), he or she is likely to buy a pharmacology book, so the confidence is high. \\
(d) Sweather and fan: Almost no one will buy sweathers and fans together, so the support is low. When a person buys a sweather, he or she is very not likely to buy a fan, so the confidence is low.


\section*{Question 4}
In a two-class, two-action problem, if the lost function is $\lambda_{11}=\lambda_{22}=0$, $\lambda_{12}=8$, and $\lambda_{21}=4$, write 
the optimal decision rule. How does the rule change if we add a third action of reject with $\lambda_1$?
\subsubsection*{Answer}
In order to reach the optimal decision rule, we first calculate the expected risks of two actions.
\begin{align*}
R(\alpha_1|x) &=\lambda_{11}\cdot P(C_1|x) +  \lambda_{12}\cdot P(C_2|x)\\
&=0\cdot P(C_1|x) + 8\cdot P(C_2|x)\\
&= 8\cdot P(C_2|x)\\
&= 8\cdot(1-P(C_1|x))
\end{align*}
\begin{align*}
    R(\alpha_2|x) &=\lambda_{21}\cdot P(C_1|x) +  \lambda_{22}\cdot P(C_2|x)\\
    &=4\cdot P(C_1|x) + 0\cdot P(C_2|x)\\
    &= 4\cdot P(C_1|x)
\end{align*}

\noindent In this case, we choose $\alpha_1$ if $$R(\alpha_1|x) < R(\alpha_2|x) \Rightarrow P(C_1|x)>2/3$$
Choose $\alpha_2$ if $$R(\alpha_1|x) > R(\alpha_2|x) \Rightarrow P(C_1|x)<2/3$$


\noindent If we add a third action of reject with $\lambda_1$, we choose $\alpha_1$ if $$R(\alpha_1) < \lambda_1 \Rightarrow P(C_1|x)>1-\frac{\lambda_1}{8}$$
we choose $\alpha_2$ if $$R(\alpha_2) < \lambda_1 \Rightarrow P(C_1|x)<\frac{\lambda_1}{4}$$
We reject when $$1-\frac{\lambda_1}{8}<P(C_1|x)<\frac{\lambda_1}{4}$$

\section*{Question 5}
Provide three examples of machine learning applications to biological or biomedical data sets. Citations and brief descriptions of the references are required in the report. 

\subsubsection*{Answer}
Machine learning has been widely used in biological studies for prediction and discovery.

\begin{enumerate}
    \item Machine learning can be used to predict DNA accessibility based on DNA sequence data and histone modification information. For example, Kelly et al. (2016) build an open-source package Basset to apply CNNs to learn the functional activity of DNA sequences from genomics data. 
    \item Machine learning can be used to predict urinary stones in CT scans. For example, Babajide et al. (2022) construct a neural network for the identification and measurement of urinary stones on NCCT images. The model reaches 100\% specificity and 100\% sensitivity.
    \item Machine learning approaches have also been used to predict breast cancer therapy response. Sammut et al. (2021) collected clinical, digital pathology, genomic and transcriptomic profiles of pre-treatment biopsies of breast tumours from 168 patients treated with chemotherapy with or without HER2 (encoded by ERBB2)-targeted therapy before surgery. Based on these data, their model has excellent performance with AUC equals 0.87.
\end{enumerate}

References:
\begin{itemize}
    \item Kelley, D. R., Snoek, J., \& Rinn, J. L. (2016). Basset: learning the regulatory code of the accessible genome with deep convolutional neural networks. Genome research, 26(7), 990–999. https://doi.org/10.1101/gr.2005/35.115
    \item Babajide, R., Lembrikova, K., Ziemba, J., Ding, J., Li, Y., Fermin, A. S., ... \& Tasian, G. E. (2022). Automated Machine Learning Segmentation and Measurement of Urinary Stones on CT Scan. Urology.
    \item Sammut, SJ., Crispin-Ortuzar, M., Chin, SF. et al.  (2022). Multi-omic machine learning predictor of breast cancer therapy response. Nature 601, 623–629. https://doi.org/10.1038/s41586-021-04278-5
\end{itemize}

\end{document}
